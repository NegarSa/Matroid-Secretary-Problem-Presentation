\documentclass[14pt]{article} 
\usepackage[utf8]{inputenc} 

%%% PAGE DIMENSIONS
\usepackage{geometry} 
\geometry{a4paper}
 \geometry{margin=4cm} % for example, change the margins to 2 inches all round
% \geometry{landscape} % set up the page for landscape

\usepackage{graphicx} 
%usepackage[parfill]{parskip} % Activate to begin paragraphs with an empty line rather than an indent
\title{\textbf{Project Report} }

\author{Negar Sakhaei \\ 9528003}
\date{} 

\begin{document}
\maketitle

\begin{Large}
The Secretary Problem presentation is prepared in three sections. First, there is an introduction section to cover the prerequisites for understanding the problem, information about online algorithms and their analysis. 
\\
\par
The second section is about the standard secretary problem, its optimal strategy, applications and shortcomings. After the problem definition, a proof is presented for the 1/e rule and its optimality. Then, the more complicated variations of this problem and the results of their best algorithms are discussed. At last, there is a brief version closer to game theory called the Googol Game is covered, alongside its minmax strategies in different settings.
\\
\par
 The last section covers the Matroid Secretary Problem in different models. At the beginning, there is a small introduction to matroids and their jargon. Then the problem is defined in its general form and some of the results obtained are presented. The original algorithms and their rigorous proofs are dropped for the sake of the time; However, there is a section about the matroid domains and their connection to variations of the main problem.
 \\
 \par
\end{Large}
\end{document}
